% !TeX encoding = UTF-8
\section{Introduction}
\textit{Graph coloring problem} is one of most famous \textit{NP-complete} problem. Given a graph $G = (V, E)$ and $k$, where $V$ is the set of vertices in the graph, $E$ is the set of edges and $k$ is the number of available colors to be assigned to vertices. The problem is to assign a certain color to every vertex $v \in V$ to a few constraints: There is no pairs of vertices which are adjacent that have the same assigned color.
This problem has pretty much applications:
\begin{itemize}
    \item[(1)] \textit{Map Coloring:} Coloring geographical maps of countries or states where no two adjacent countries cannot be assigned same color. Four colors are enough to color.
    \item[(2)] \textit{Sudoku:} Sudoku is a variation of graph coloring problem where every cell represents a vertex. There is an edge between two vertices if they are in same row or same column or same block.
    \item[(3)] \textit{Register Allocation:} In compiler optimization, register allocation is the process of assigning a large number of target program variables onto a small number of CPU registers.\cite{example}
\end{itemize}

\begin{claim}
2-COLORING problem is the simplest among all coloring problems, because it is solvable in polynomial time. (2-COLORING $\in$ $P$)
\end{claim}

\begin{proof}
Suppose the given graph $G = (V, E)$ and color 1, 2. 
\begin{itemize}
    \item[(1)] Randomly pick $v \in V$, color it with 1 and record the number of coloring and the assigned colors.
    \item[(2)] Apply BFS starting with vertex $v$ and color its neighbors with the other color 2 and so on. The point is that for $\forall u \in V$, we should color its neighbor with the other color alternatively.
    \item[(3)] After finishing BFS, go through all vertices again to check whether there is a vertex that is assigned with different colors other than its neighbor(s). If \textit{yes}, then the graph is \textit{not} 2-colorable, \textit{otherwise} it is 2-colorable.
\end{itemize}
The running time is similar to BFS: $\mathcal{O}(|V| + |E|)$
\end{proof}

\begin{observation}
We can check whether a graph is bipartite by coloring the graph using two colors. If a given graph is 2-colorable, then it is Bipartite.
\end{observation}

\begin{observation}
$k$-COLORING problems are $NP-complete$ for $k \geq 3$ and 3-COLORING $\le_p$ k-COLORING. 
\end{observation}

As previously stated, the k-COLORING problem is NP-complete, which means that it seems hardly possible to have a polynomial time algorithm for this problem. Moving on now to consider 3-COLORING problem under circumstance that there is no triangle within the given graph $G$. Grötzsch’s theorem states that each triangle-free planar graph is 3-colorable. Thomassen \cite{Thomassen1994Grtzschs3T} has also found two proofs and extended this result in various way, by which an quadratic algorithm for finding suitable 3-coloring can be developed possibly. Kowalik \cite{article} maintains a data structure called \textit{Short Path Data Structure (SPDS)}, which will be built in linear time and enables that finding shortest paths of length at most 2 in planar graph takes $\mathcal{O}(1)$ time. The SPDS will be constantly updated during the particular sequence of operations. And its running time is $\mathcal{O}(n\log{}n)$. After that, Dvorak, Kawaravayashi and Thomas \cite{dvorak2013threecoloring} have designed a linear-time algorithm which still replies on the Grötzsch’s theorem but avoids complex data structures. Nevertheless, their paper is quite complicated for readers. So this paper will give a deeper and detailed view into their paper for better understanding of their main idea and proofs. 
